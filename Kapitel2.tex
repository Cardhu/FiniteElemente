\section{Ritz-Galerkin-Methode}
Bisher haben wir Finite-Differenzen-Methoden(FDM) betrachtet.
\begin{bemerkung}[Probleme mit FDM]
\begin{itemize}
\item Wir nutzen reguläre Gitter, dies kann zu Problemen bei kurvigem Rand und hoher Ordnung führen.
\item $a$ ist eventuell unstetig.
\item Wir erhalten eventuell nichtreguläre Lösungen ($u\notin C^4$!)
\end{itemize}
\subsection{Das diskrete Problem}
$V_N \sqsubset V$ mit $\dim(V_N)=N \in \MdN$\\
Konformer Ansatz im Gegensatz zum nicht-konformen Ansatz.\\
$V_N \sqsubset V$. Im Folgenden sei immer $u^D = 0$.\\
Diskretes Problem: Suche $u_N \in V_N$ mit 
\[A(u_N, u_N) = F[v_N] \forall v_N \in V_N\]
Mit Vorraussetzungen wie an $A$, $F$ wie vorher folgt die Existenz von $u_N$ nach Lax-Milgram. Außerdem finden wir
\[||u_N||_{H^1(\Omega)} \leq \underbrace{\frac{\alpha_F}{\alpha_0}}_{\text{unabhängig von }N}.\]
Sei ${\psi_i}_{i \in {1,...,N}}$ eine Basis von $V_N$. D.h. es existiert eine eindutige Darstellung
\[u_N = \sum_{j=1}^{N}\alpha_j \psi_j \]
mit $\alpha_j \in \MdR$. Einsetzen ergibt
\[\sum_{j=1}^{N}\alpha_j A [\psi_j, \psi_i] = F(\psi_i)\] 
für $i=1,\ldots , N$. $\vec{A}\in\MdR^{N\times N}$, $\vec{A}=[A[\psi_j, \psi_i]]_{ij}$
\[\vec{u} = [x_j]_j,\quad \vec{F}=[F[\psi_j]]_j, \quad \vec{A}\vec{u}=\vec{F} \]
Dabei handelt es sich für symetrisches $A[\cdot, \cdot]$ um ein lineares und positiv definites Gleichungssystem.
\begin{definition}[Energienorm]
Für positiv definites $A$ definiert man 
\[||v||_E := \left( A[v,v]\right)^{\frac{1}{2}}\]
\subsection{Das Lemma von Cea}
\begin{theorem}[Cea-Lemma] $A$,$F$ wie in Theorem 3 und $u_N$ die jeweiligen Lösungen des kontinuierlichen beziehungsweise diskreten Problems. Dann
\[||u - u_N||_E = \inf_{v_N \in V_N} ||u-v_N||_E\]
D.h. $u_N$ ist bestmöglich in $V_N$ in der $||\cdot ||_E$-Topologie.
\end{theorem}
\begin{beweis} $V_N \in V_N$ beliebig. 
\begin{equation}
A[u-u_N]=F[v_N]-F[v_N] = 0
\end{equation}
Diese Eigenschaft nennt man auch \begriff{Galerkin-Orthogonalität} des Fehlers.\\
\begin{eqnarray*}
||u-u_N||^2_E &= A[u-u_N, u-u_N] \stackrel{\text{(1)}}{=} A[u-u_N, u-n_N]\\
&\stackrel{\text{CSU}}{\leq} ||u-u_N||_E||u-v_N||_E\\
\rightarrow ||u-u_N||_E \leq ||u-v_N||_E \forall v_N \in V_N\\
\end{eqnarray*}
Abschließend erfolgt ein Grenzübergang zum Infimum.
\end{beweis}
\begin{korollar}[Konvergenz]
Annahme: Es existiert ein linearer stetiger Operator.:
\[P_N: V\to V_N: ||v-P_N v ||_V \to 0 \quad (N \to \infty)\]
für jedes feste $v \in V$. Dann gilt
\[||u-u_N||_V \to 0 \quad N \to \infty\]
\end{korollar}
\begin{beweis}[Koerzivität]
\begin{eqnarray*}\alpha_0||u-u_N||_V^2 &\leq ||u-u_N||_E^2 \\
&= \inf_{v_N\in V_N} ||u-v_N||_E^2 \leq ||u - \underbrace{P_Nu}_{\in V_N}||_E^2 \\ &= \alpha_1 ||u - P_Nu||_V^2 \stackrel{N \to \infty}{\to} 0\end{eqnarray}\\
\end{beweis}
\begin{bemerkung}[Quasi-Optimalität]
Im Allgemeinen erhält man das Resultat
\[ ||u-u_N||_V \leq C_S \inf_{v_N \in V_N}||u-v_N||_V \]
mit einem $C_S > 0$. Diese eigenschaft bezeichnet man als \begriff{Quasi-Optimalität} und $C_S$ als \begriff{Stabilitätskonstante}. Von Konvergenz sprechen wir in diesem Zusammenhang dann, wenn Stabilität und Approximierbarkeit gegeben sind. Dabei ist wichtig, dass von Stabilität nur gesprochen wird, wenn $C_S$ nicht von $N$ abhängt.
\end{bemerkung}
Wir werden $P_N$ explizit konstruieren. Typischerweise gilt
\[u - P_N u ||_V \leq C_I N^{- \kappa}\]
für ein $\kappa>0$. $C_I$ hängt dabei von $u$ ab. 
Im Allgemeinen gilt
\[u \in \tilde{V} \sqsubset V = H_0^1(\Omega)\]
Oft, aber nicht immer, gilt außerdem
\[\tilde{V} = H^2(\Omega)\cap H_0^1(\Omega)\]
$u_N$  ist hier immer die exakte Lösung des Gleichungssystems. $\vec{A_{ij}}$, $\vec{F_j}$ benötigen exakte Integration. \\
\begin{paragraph}[Standpunkt:] Die Fehler der Grundlegenden Numerik seien geringer als die Verfahrensfehler in den hier vorgestellten Schritten. 
\end{paragraph}
\begin{paragraph}[Annahme:] das diskrete Problem sei 
\[A_N[u_N, v_N] = F_N[v_N] \forall v_N \in V_N\]
Im Folgenden muss $V_N \sqsubset V$ muss im Folgenden nicht gelten.
\end{paragraph}
\subsection{Das Strang-Lemma}
\begin{lemma}[Das Strang-Lemma]
Seien $A,F$ wie in Theorem 3, $A_N$ symetrisch positiv definit, $F_N$ stetige Linearform. $A_N$ sei definiert auf $V + V_N$ mit 
\[\beta_0 A[v_N, v_N] \leq A_N[v_n, v_N] \quad \forall v_N \in V_N\]
für ein $\beta_0 > 0$. Dann gilt:
