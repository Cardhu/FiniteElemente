\chapter{Variationsgleichung}
\section{Modellierung mit partiellen Dfferentialgleichungen}
\subsubsection{W�rmeleitungsgleichung}
Wir betrachten einen W�rmeleiter, der am einen Ende die Temperatur $T_1$ und am anderen Ende die Temperatur $T_2$ hat. O.B.d.A. $T_1 > T_2$. Dann flie�t W�rme von 1 nach 2. Sei weiter $\Omega$ ein Gebiet,
\[ u: (0,T)\times \overline{\Omega} \rightarrow \MdR_{>0} \]
sei die Temperatur abh�ngig von Zeit und Ort. Eine Temperaturdifferenz erzeugt einen W�rmefluss $q=-a \nabla u$ ($a>0$ Materialkonstante W�rmeleitf�higkeit). Es ergibt sich die Bilanzgleichung
\[ \frac{\mathrm d}{\mathrm dt} \int_V \rho = \int_{\partial V} q \cdot n\]
wobei $V$ das Volumen, $\rho$ die Dichte, $q$ der W�rmefluss und $n$ die �u�ere Normale ist. Es folgt
\[\int_V \left( \partial_t \rho + \nabla \cdot q\right) = 0 \quad \forall V \subset \Omega\]
und daraus, da das Integral f�r beliebige Gebiete $V$ gilt, auch
\[\partial_t \rho + \nabla \cdot q = 0\]
In oben beschriebenen Fall gelten $\rho = u$ und $q = -a \grad u$ also
\[ \partial_t u - \nabla\cdot (a\nabla u) = 0 \]
in $\Omega$. Wir erwarten einen Temperaturausgleich f�r gro�e Zeiten, also einen station�ren Zustand f�r $t \to \infty$, d.h. $\partial_t u \to 0$. Eingesetzt finden wir
\begin{eqnarray*}
- \nabla (a \nabla u) & = 0 & \text{ in } \Omega\\
u & = u^D & \text{ auf } \partial \Omega\\
-a \Delta u & = 0 & \text{ (falls } a\equiv \mathrm const \text{ )}
\end{eqnarray*}
\paragraph{'Quantity of Interest':}
$\int_W q \cdot n$ wobei $W \subset \partial \Omega$ das Randst�ck mit interessantem W�rmefluss ist. Die 'Quantity of Interest' beschreibt einen W�rmestrom �ber einen Teil des Randes.
\subsubsection{Elektrostatik}
Wir bezeichnen mit $\rho$ die Ladungsdichte, die ein elektrisches Feld $E$ verursacht. Aus den Maxwellgleichungen folgt:
\[ -\nabla \cdot (aE) = \rho \]
wobei $a$ (in der Physik $\epsilon$) die Permittivit�t darstellt. Oft fordert man ein "`Wirbelfreies"' elektrisches Feld, also $\mathrm rot (E) = \nabla \times E = 0$. Daraus ergibt sich 
\begin{eqnarray*}
\exists u: E & = & -\nabla u\\
- \nabla \cdot (a \nabla u) & = & \rho \text{ in } \Omega \\
u & = & u^D \text{ auf } \partial \Omega
\end{eqnarray*}
Im Fall $\partial \Omega = \Gamma^D \dot{\cup} \Gamma^N$ setzen wir\\
\[\begin{cases}u^D & \text{ auf } \Gamma^D \\ an \cdot \nabla u \equiv a \cdot \partial_n u = 0 & \text{ auf } \Gamma^N \end{cases}\] 
\subsubsection{Str�mung durch por�se Medien}
Sei $p$ eine Druckverteilung, $u: \overline{\Omega}\rightarrow\MdR$. \textit{Darcy's law} besagt
\[ q = -\kappa (\nabla p + \rho g \vec{i_z}) \] 
wobei $\kappa$ eine Materialkonstante ist. $f$ beschreibe die Quellen und Senken der Str�mung. 
\[ - \nabla \cdot (\kappa (\nabla p + \rho g \vec{i_z})) = f \quad \text{ in } \Omega\]
Wir definieren die \begriff{piezometrische H�he}: $u = \frac{p}{\rho g} + x_z$. Wir k�nnen das Problem nun in zwei Varianten beschreiben: \begin{enumerate}
	\item \textbf{Ohne S�ttigung:} \[ -\nabla \cdot (\kappa \nabla u) = f \] in $\Omega$ mit Randbedingungen und
	\item \textbf{Mit S�ttigung:} \[ -\nabla \cdot (\kappa(u) \nabla u) = f \] 
	Diese Form ist nicht linear und beschreibt die sich f�llenden Poren durch die ver�nderliche Materialeigenschaft $\kappa(u)$.
\end{enumerate}

\subsubsection{Station�re Wellen}
Elektromagnetische Wellen bzw. elektrostatische Membranen erf�llen die Wellengleichung, k�nnen also wie folgt beschrieben werden:
\[\Phi: (0,T) \times \overline{\Omega} \rightarrow \MdR, \partial_t^2 \Phi - \nabla\cdot(c^2\nabla\Phi) = 0\] in $\Omega$ mit entsprechenden Randbedingungen. Im Falle einer eingespannten Membran ist diese Randbedingung zum Beispiel konstant $0$. Nun schr�nken wir den L�sungsraum ein, betrachten also einen Spezialfall. Wir suchen ausschlie�lich L�sungen der Form
\[ \Phi(t,x) = u(x)\exp^{i\omega t} \quad \omega \in \MdR, \Re(\Phi), \Im(\Phi)\in \MdR\]
Durch Einsetzen finden wir
\[-\omega^2 u - \nabla\cdot(c^2 \nabla u) = 0\]
Im Fall einer eingespannten Membran und konstanter Wellengeschwindigkeit $c$ ergibt sich zum Beispiel folgendes Problem:
\[\begin{cases}-\Delta u = \left( \frac{\omega}{c}\right)^2u & \quad \text{ in } \Omega \\ u=0 & \quad \text{ auf }\partial \Omega \end{cases}\]
Gesucht sind dabei $\omega , u \neq 0$ die das Problem l�sen, es handelt sich also um ein Eigenwertproblem. 

\subsubsection{Reaktionsgleichungen}
Chemische und biologische Reaktionen lassen sich oft folgenderma�en beschreiben: 
\[u: \overline{\Omega} \rightarrow \MdR \]
\begin{eqnarray*}
-\Delta u & = \exp^{-\frac{\lambda}{u}} \quad \text{ in } \Omega
 u &= u_0 > 0 \quad \text{ auf } \partial \Omega
\end{eqnarray*}
wobei $u$ die Temperatur des ablaufenden Prozesses beschreibt. 

\subsubsection{Minimalfl�chengleichung}
$u: \overline{\Omega} \rightarrow \MdR$, $\Omega \subset \MdR^2$, $F: \Omega \rightarrow \MdR^3, F:[x_1, x_2] \to [x_1, x_2, u(x_1, x_2)]$
\begin{eqnarray*}
A(u) &= \int_{\Omega} | \partial_1 F\times \partial_2 F | \\
&= \int_{\Omega} \sqrt{1 + |\nabla u|^2}
\end{eqnarray*} 
mit Randbedingungen der Form $u(x) = g(x) \forall x \in \partial \Omega$. Ein Minimum erf�llt dann die Bedingung
\[- \nabla \cdot \underbrace{\left( \frac{\nabla u}{1 + |\nabla u|^2}\right)}_{= \kappa(\nabla u) \nabla u} = 0 \quad \text{ in }\Omega\] 
Dabei ergibt sich das unangenehme Ph�nomen, dass $\nabla u$ in jedem Punkt $\infty$ sein kann, obwohl $u$ bekannt ist. Ein Beispiel hierf�r ist eine Halbkugel.

\subsubsection{Inkompressible Navier-Stokes-Gleichungen}
Modellierung: Transportterm
\[ q = \underbrace{-\nu \nabla u}_{\text{Diffusion}} + \underbrace{bu}_{\text{Transport mit dem Fluss }b: \Omega \rightarrow \MdR^3} \]
\[- \nabla \cdot (\nu \nabla u) + \nabla \cdot(bu) = 0\]
Inkompressibel bedeutet $\nabla \cdot b = 0$.\\
\[\Rightarrow - \nabla \cdot (\nu \cdot \nabla u)+ b\cdot \nabla u = 0\]
Da diese Gleichung einen Transport und einen Diffusionsterm enth�lt nennt man sie auch die \begriff{Transport-Diffusions-Gleichung}. Navier-Stokes (f�r konstantes $\nu$):
\begin{eqnarray*}
\partial_t u - \nu \Delta u + u \cdot \nabla u &= f \quad \text{ in } \Omega\\
\nabla \cdot u &= 0 \quad \text{ in } \Omega
\end{eqnarray*}
mit entsprechenden Rand- und Anfangsbedingungen.

\section{Das Dirichlet-Randwertproblem}
Sei $\Omega \subset \MdR^d$ offen und beschr�nkt. Seien weiter $a,c,f: \overline{\Omega}\rightarrow\MdR$ und $u^D: \partial \Omega \rightarrow \MdR$. Das \begriff{lineare Randwertproblem} in Divergenzform lautet:
\begin{eqnarray}
-\nabla ( a\nabla u) + cu &=f \text{ in } \Omega\\
u &= u^D \text{ auf }\partial \Omega
\label{lineares RWP}
\end{eqnarray}
der Spezialfall $a=1, c=0$ f�hrt auf 
\begin{eqnarray}
-\Delta u &= f \text{ in } \Omega\\
u &= u^D \text{ auf } \partial \Omega
\label{Poissonproblem}
\end{eqnarray}
das \begriff{Poisson-Problem}.\\
Die Vorgabe von $u=u^D$ auf $\partial \Omega$ hei�t \begriff{Dirichlet-Problem}. Man kann eine schwache Form wie folgt herleiten: Multipliziere die Differentialgleichung mit einer Testfunktion $v \in$ 