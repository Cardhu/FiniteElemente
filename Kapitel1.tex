\chapter{Variationsgleichung}
\section{Modellierung mit partiellen Dfferentialgleichungen}
\subsubsection{W�rmeleitungsgleichung}
Wir betrachten einen W�rmeleiter, der am einen Ende die Temperatur $T_1$ und am anderen Ende die Temperatur $T_2$ hat. O.B.d.A. $T_1 > T_2$. Dann flie�t W�rme von 1 nach 2. Sei weiter $\Omega$ ein Gebiet,
\[ u: (0,T)\times \overline{\Omega} \rightarrow \MdR_{>0} \]
sei die Temperatur abh�ngig von Zeit und Ort. Eine Temperaturdifferenz erzeugt einen W�rmefluss $q=-a \nabla u$ ($a>0$ Materialkonstante W�rmeleitf�higkeit). Es ergibt sich die Bilanzgleichung
\[ \frac{\mathrm d}{\mathrm dt} \int_V \rho = \int_{\partial V} q \cdot n\]
wobei $V$ das Volumen, $\rho$ die Dichte, $q$ der W�rmefluss und $n$ die �u�ere Normale ist. Es folgt
\[\int_V \left( \partial_t \rho + \nabla \cdot q\right) = 0 \quad \forall V \subset \Omega\]
und daraus, da das Integral f�r beliebige Gebiete $V$ gilt, auch
\[\partial_t \rho + \nabla \cdot q = 0\]
In oben beschriebenen Fall gelten $\rho = u$ und $q = -a \grad u$ also
\[ \partial_t u - \nabla\cdot (a\nabla u) = 0 \]
in $\Omega$. Wir erwarten einen Temperaturausgleich f�r gro�e Zeiten, also einen station�ren Zustand f�r $t \to \infty$, d.h. $\partial_t u \to 0$. Eingesetzt finden wir
\begin{eqnarray*}
- \nabla (a \nabla u) & = 0 & \text{ in } \Omega\\
u & = u^D & \text{ auf } \partial \Omega\\
-a \Delta u & = 0 & \text{ (falls } a\equiv \mathrm const \text{ )}
\end{eqnarray*}
\paragraph{'Quantity of Interest':}
$\int_W q \cdot n$ wobei $W \subset \partial \Omega$ das Randst�ck mit interessantem W�rmefluss ist. Die 'Quantity of Interest' beschreibt einen W�rmestrom �ber einen Teil des Randes.
\subsection{Elektrostatik}
Wir bezeichnen mit $\rho$ die Ladungsdichte, die ein elektrisches Feld $E$ verursacht. Aus den Maxwellgleichungen folgt:
\[ -\nabla \cdot (aE) = \rho \]
wobei $a$ (in der Physik $\epsilon$) die Permittivit�t darstellt. Oft fordert man ein "`Wirbelfreies"' elektrisches Feld, also $\mathrm rot (E) = \nabla \times E = 0$. Daraus ergibt sich 
\begin{eqnarray*}
\exists u: E & = & -\nabla u\\
- \nabla \cdot (a \nabla u) & = & \rho \text{ in } \Omega \\
u & = & u^D \text{ auf } \partial \Omega
\end{eqnarray*}
Im Fall $\partial \Omega = \Gamma^D \dot{\cup} \Gamma^N$ setzen wir\\
\[\begin{cases}u^D & \text{ auf } \Gamma^D \\ an \cdot \nabla u \equiv a \cdot \partial_n u = 0 & \text{ auf } \Gamma^N \end{cases}\] 
